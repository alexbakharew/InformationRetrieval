\CWHeader{Лабораторная работа \textnumero 1 \enquote{Добыча корпуса документов}}

Необходимо подготовить корпус документов, который будет использован при выполнении остальных лабораторных работ:
\begin{itemize}
    \item Скачать его к себе на компьютер. В отчёте нужно указать источник данных.
    \item Ознакомиться с ним, изучить его характеристики. Из чего состоит текст? Есть ли
дополнительная мета-информация? Если разметка текста, какая она?
    \item Разбить на документы.
    \item Выделить текст.
    \item Найти существующие поисковики, которые уже можно использовать для поиска по
выбранному набору документов (встроенный поиск Википедии, поиск Google с использованием ограничений на URL или на сайт). Если такого поиска найти невозможно, то использовать корпус для выполнения лабораторных работ нельзя!
    \item Привести несколько примеров запросов к существующим поисковикам, указать недостатки в полученной поисковой выдаче.
\end{itemize}

В результатах работы должна быть указаны статистическая информация о корпусе:
\begin{itemize}
    \item Размер \enquote{сырых} данных.
    \item Количество документов.
    \item Размер текста, выделенного из \enquote{сырых} данных.
    \item Средний размер документа, средний объём текста в документе.
\end{itemize}

\pagebreak
