\CWHeader{Лабораторная работа \textnumero 1 \enquote{Добыча корпуса документов}}
Нужно реализовать ввод поисковых запросов и их выполнение над индексом, получение
поисковой выдачи.
Синтаксис поисковых запросов:

\begin{itemize}
\item Размер \enquote{сырых} данных.
\item Пробел или два амперсанда, «\&\&», соответствуют логической операции «И».
\item Две вертикальных «палочки», «||» – логическая операция «ИЛИ»
\item Восклицательный знак, «!» – логическая операция «НЕТ»
\item Могут использоваться скобки.
\end{itemize}

Парсер поисковых запросов должен быть устойчив к переменному числу пробелов, максимально
толерантен к введённому поисковому запросу.

Примеры запросов:
\begin{itemize}
\item [ московский авиационный институт ]
\item  [ (красный || желтый) автомобиль ]
\item [ руки !ноги ]
\end{itemize}

Для демонстрации работы поисковой системы должен быть реализован веб-сервис, реализующий
базовую функциональность поиска из двух страниц:
\begin{itemize}
    \item  Начальная страница с формой ввода поискового запроса.
    \item  Страница поисковой выдачи, содержащая в себе форму ввода поискового запроса, 50 результатов поиска в виде текстов заголовков документов и ссылок на эти документы, а так же ссылку на получение следующих 50 результатов.
\end{itemize}
Так же должна быть реализована утилита командной строки, загружающая индекс и
выполняющая поиск по нему для каждого запроса на отдельной строчке входного файла.
В отчёте должно быть отмечено:
\begin{itemize}
    \item  Скорость выполнения поисковых запросов.
    \item  Примеры сложных поисковых запросов, вызывающих длительную работу.
    \item  Каким образом тестировалась корректность поисковой выдачи
\end{itemize}

\pagebreak
