\CWHeader{Лабораторная работа \textnumero 3 \enquote{Булев индекс}}

Требуется построить поисковый индекс, пригодный для булева поиска, по подготовленному в ЛР1
корпусу документов.
Требования к индексу:

\begin{itemize}
    \item Самостоятельно разработанный, бинарный формат представления данных. Формат
необходимо описать в отчёте, в побайтовом (или побитовом) представлении.
	\item Формат должен предполагать расширение, т.к. в следующих работах он будет меняться
под требования новых лабораторных работ.
	\item Использование текстового представления или готовых баз данных не допускается.
	\item Кроме обратного индекса, должен быть создан «прямой» индекс, содержащий в себе как
минимум заголовки документов и ссылки на них (понадобятся для выполнения ЛР4, при
генерации страницы поисковой выдачи).
	\item Для термов должна быть как минимум понижена капитализация. 
\end{itemize}

В отчёте должно быть отмечено как минимум:
\begin{itemize}
	\item Выбранное внутренне представление документов после токенизации.
	\item Выбранный метод сортировки, его достоинства и недостатки для задачи индексации.
\end{itemize}

Среди результатов и выводов работы нужно указать:
\begin{itemize}
\item Количество термов.
\item Средняя длина терма. Сравнить со средней длинной токена, вычисленной в ЛР1 по курсу ОТЕЯ. Объяснить причину отличий. 
\item Скорость индексации: общую, в расчёте на один документ, на килобайт текста.
\item Оптимальна ли работа индексации? Что можно ускорить? Каким образом? Чем она
ограниченна? Что произойдёт, если объём входных данных увеличится в 10 раз, в 100 раз,
в 1000 раз?
\end{itemize}


\pagebreak
